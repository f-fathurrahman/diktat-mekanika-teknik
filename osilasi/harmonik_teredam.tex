\section{Osilasi harmonik teredam}

Misalkan sekarang selain dari gaya pemulih juga
bekerja gaya gesek atau gaya redam, yang disebabkan misalnya oleh
gesekan dengan udara atau fluida lain.
Gaya redam ini biasanya dimodelkan sebagai
\begin{equation}
F_{\mathrm{damp}} = -bv = -b\frac{\mathrm{d}x}{\mathrm{d}t}
\end{equation}
dengan $b$ adalah suatu konstanta.
Pada kasus ini, persamaan gerak dari osilator menjadi sebagai berikut:
\begin{equation}
m\frac{\mathrm{d}^{2}x}{\mathrm{d}t^{2}}+b\frac{\mathrm{d}x}{\mathrm{d}t}+kx=0
\end{equation}
Dengan mendefinisikan konstanta $\beta$, disebut konstanta redaman, \emph{damping constant}
sebagai berikut
\begin{equation}
\beta = \frac{b}{2m}
\end{equation}
dan frekuensi alami (\emph{natural frequency})
\begin{equation}
\omega_{0}=\sqrt{\frac{k}{m}},
\end{equation}
persamaan osilator teredam dapat ditulis menjadi
\begin{equation}
\frac{\mathrm{d}^{2}x}{\mathrm{d}t^{2}} + 2\beta \frac{\mathrm{d}x}{\mathrm{d}t} + \omega_{0}^{2}x = 0
\label{eq:Taylor_5_28}
\end{equation}

Persamaan ini masih merupakan persamaan diferensial orde dua, linear,
dan homogen. Jika kita dapat menemukan dua solusi independen $x_{1}(t)$
dan $x_{2}(t)$ pada persamaan ini, maka solusi umum dari persamaan
ini dapat dituliskan sebagai superposisi dari dua solusi, yaitu $C_{1}x_{1}(t)+C_{2}x_{2}(t)$.

Seperti kasus osilator harmonik tanpa redaman,
\textit{ansatz} solusi coba yang sering digunakan untuk persamaan diferensial ini adalah
\begin{equation*}
x(t) = e^{rt}
\end{equation*}
Dapat dihitung
\begin{equation*}
\frac{\mathrm{d}x}{\mathrm{d}t} = re^{rt}
\end{equation*}
dan
\begin{equation*}
\frac{\mathrm{d}^{2}x}{\mathrm{d}t^{2}} = r^{2}e^{rt}
\end{equation*}
Dengan melakukan substitusi pada persamaan diferensial untuk osilasi
teredam diperoleh persamaan bantu (\textit{auxiliary equation}) dalam $r$:
\begin{equation}
r^{2} + 2\beta r + \omega_{0}^{2} = 0
\end{equation}
Persamaan ini adalah persamaan kuadrat dengan akar-akar sebagai berikut:
\begin{align*}
r_{1} & =\frac{-2\beta + \sqrt{(2\beta)^{2} - 4(1)(\omega_{0}^{2})}}{2(1)} = -\beta+\sqrt{\beta^{2}-\omega_{0}^{2}}\\
r_{2} & =\frac{-2\beta - \sqrt{(2\beta)^{2} - 4(1)(\omega_{0}^{2})}}{2(1)} = -\beta-\sqrt{\beta^{2}-\omega_{0}^{2}}
\end{align*}
Maka dengan menyelesaikan persamaan bantu, kita mendapatkan dua fungsi $e^{r_{1}t}$ dan $e^{r_{2}t}$
yang merupakan dua solusi independen. 
Solusi umumnya dari Persamaan \eqref{eq:Taylor_5_28} adalah:
\begin{align*}
x(t) & = C_{1}e^{r_{1}t}+C_{2}e^{r_{2}t} \\
 & = e^{-\beta t} \left( C_1 e^{\sqrt{\beta^2 - \omega_0} t} + C_2 e^{ -\sqrt{\beta^2 - \omega_0} t} \right)
\end{align*}

Sekarang, bergantung dari nilai-nilai $r_1$ dan $r_2$ (atau parameter $\beta$ dan $\omega_0$),
kita akan meninjau beberapa kasus yang mungkin.


\subsection*{Osilasi tidak teredam}

Kasus pertama yang akan ditinjau adalah ketika $\beta = 0$, atau tidak ada redaman sama sekali.
Kasus pada kasus ini $r_1$ dan $r_2$ akan bernilai imajiner murni $r_{1,2} = \pm \omega_0$ yang
sudah dibahas pada bagian sebelumnya.

Untuk selanjutnya kita akan mengasumsikan nilai $\beta > 0$.

\subsection*{Osilasi teredam lemah (underdamped)}

Pada kasus $\beta > 0$ dan $\beta < \omega_{0}$ akar-akar $r_{1}$ dan $r_{2}$ akan
bernilai kompleks karena kuantitas $\beta**2 - \omega_0^2$ akan bernilai negatif.
Kita dapat menuliskan
\begin{equation*}
\sqrt{\beta^{2}  -\omega_{0}^{2}} = \mathrm{i} \sqrt{\omega_{0}^{2}-\beta^{2}} = \mathrm{i}\omega_{1}
\end{equation*}
dengan 
\begin{equation*}
\omega_{1} = \sqrt{\omega_{0}^{2} - \beta^{2}}
\end{equation*}
Kasus ini juga sering disebut sebagai kasus \textit{underdamped}.

Solusi $x(t)$ dapat ditulis sebagai berikut
\begin{equation}
x(t) = e^{-\beta t} \left( C_{1}e^{\mathrm{i}\omega_{1}t} + C_{2}e^{-\mathrm{i}\omega_{1}t} \right)
\end{equation}
Karena bentuk eksponensial di dalam tanda kurung dapat diubah menjadi fungsi cosinus
dengan pergeseran fasa, maka juga dapat ditulis
\begin{equation}
x(t) = Ae^{-\beta t}\cos(\omega_{1}t - \delta)
\end{equation}


\subsection*{Osilasi teredam kuat (overdamped)}

Tinjau kasus di mana $\beta > \omega_{0}$ sehingga akar-akar $r_{1}$ dan $r_{2}$ bernilai real.

Solusi $x(t)$ dapat ditulis menjadi:
\begin{equation}
x(t) = C_{1}e^{ - \left( \beta - \sqrt{\beta^{2}-\omega_{0}^{2}} \right) t} +
       C_{2}e^{ - \left( \beta + \sqrt{\beta^{2}-\omega_{0}^{2}} \right) t}
\end{equation}
Kasus ini juga sering disebut sebagai \textit{overdamped}.
     
Perhatikan bahwa kedua pangkat dari eksponensial tersebut adalah negatif, sehingga $x(t)$ akan menuju 0 ketika
$t \rightarrow \infty$.
Perhatikan juga bahwa suku pertama akan meluruh lebih lama daripada suku kedua karena nilai
eksponen negatif suku pertama lebih kecil daripada suku
kedua. Dalam jangka panjang suku ini akan mendominasi, karena suku kedua sudah meluruh lebih dahulu (lebih cepat
mendekati nol). Oleh karena itu laju di mana gerakan akan berhenti dapat dikarakterisasi oleh
koefisien dari ekponensial pertama yaitu
\begin{equation}
\text{parameter peluruhan} = \beta - \sqrt{\beta^2 - \omega_0^2}
\end{equation}
Perhatikan bahwa, laju peluruhan akan menjadi lebih kecil apabila konstanta $\beta$
menjadi semakin besar.


\subsection*{Osilasi teredam kritis (critically damped)}

Pada kasus ini $\beta=\omega_{0}$ sehingga
akar-akar $r_{1}$ dan $r_{2}$ bernilai sama, yaitu $r_{1} = r_{2} = \beta$.

Karena hanya ada satu solusi khusus, kita perlu mencari satu solusi khusus
lagi agar dapat menuliskan solusi umum. Solusi khusus tambahan yang
dapat digunakan adalah
\begin{equation}
x(t) = te^{-\beta t}
\end{equation}
sehingga solusi umum yang diperoleh adalah
\begin{align*}
x(t) & = C_{1}e^{-\beta t} + C_{2}te^{-\beta t} \\
     & = e^{-\beta t}\left(C_{1} + C_{2}t \right)
\end{align*}