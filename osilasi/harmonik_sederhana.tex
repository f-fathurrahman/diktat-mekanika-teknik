\section{Gerak harmonik sederhana}

Dengan menggunakan hukum Newton:
\footnote{
Notasi di buku: $\ddot{x}\equiv\dfrac{\mathrm{d}^{2}x}{\mathrm{d}t^{2}}$
}
\begin{equation*}
m \frac{\mathrm{d}^2x}{\mathrm{d}t^2} = F_x
\end{equation*}
%
diperoleh
%
\begin{equation*}
m \frac{\mathrm{d}^2x}{\mathrm{d}t^2}  = -kx
\end{equation*}
%
atau:
%
\begin{equation}
\frac{\mathrm{d}^2x}{\mathrm{d}t^2} = -\omega_0^2 x
\label{eq:Taylor_5_4}
\end{equation}
%
dengan $\omega_0$ adalah frekuensi sudut
%
\begin{equation}
\omega_{0} = \sqrt{\frac{k}{m}}
\end{equation}
%
Ingat juga hubungan frekuensi sudut dengan periode dan frekuensi linear $f$
yaitu $\omega = \dfrac{2 \pi}{T} = 2 \pi f$.

Tugas kita selanjutnya adalah mencari solusi $x(t)$ dari persamaan diferensial
\eqref{eq:Taylor_5_4}. Solusi ini dapat dituliskan dalam berbagai bentuk.



\subsection*{Solusi eksponensial}

Biasanya untuk persamaan diferensial linear, digunakan \textit{ansatz} solusi
coba (\textit{trial solution}) berupa fungsi eksponensial:
\begin{align*}
x_{1}(t) & = e^{\mathrm{i}\omega_0 t} \\
x_{2}(t) & = e^{-\mathrm{i}\omega_0 t}
\end{align*}
%
Dapat dicek bahwa kedua solusi coba tersebut memenuhi persamaan \eqref{eq:Taylor_5_4}.
Misalnya, untuk $x_{1}(t) = e^{\mathrm{i}\omega_0 t}$ diperoleh turunan pertama
%
\begin{equation*}
\frac{\mathrm{d}x_{1}(t)}{\mathrm{d}t} = \mathrm{i}\omega e^{\mathrm{i} \omega_0 t}
\end{equation*}
%
dan turunan kedua
%
\begin{equation*}
\frac{\mathrm{d}^{2}x_{1}(t)}{\mathrm{d}t^{2}} = (\mathrm{i}\omega_0)^{2} e^{\mathrm{i} \omega t} =
-\omega_{0}^{2}e^{\mathrm{i} \omega t}
\end{equation*}
sehingga:
\begin{equation*}
\frac{\mathrm{d}^{2}x_1(t)}{\mathrm{d}t^{2}} = -\omega_0^{2} x_{1}(t)
\end{equation*}
Dengan cara yang sama juga dapat ditunjukkan bahwa $x_{2}(t)$ juga
memenuhi \eqref{eq:Taylor_5_4}.

Solusi umum dari persamaan \eqref{eq:Taylor_5_4} adalah kombinasi linear dari
$x_{1}(t)$ dan $x_{2}(t)$:
\begin{equation}
x(t) = C_{1} e^{\mathrm{i}\omega_{0} t} + C_{2} e^{-\mathrm{i}\omega_{0} t}
\label{eq:Taylor_5_5}
\end{equation}
dengan $C_1$ dan $C_2$ adalah konstanta yang dapat
ditentukan dari syarat awal: $x(t=0)$ dan $\dfrac{\mathrm{d}x}{\mathrm{d}t}(t=0)$.



\subsection*{Solusi sinusoidal (sin dan cos)}

Solusi eksponensial pada Persamaan \eqref{eq:Taylor_5_5} sangat mudah untuk dimanipulasi
sehingga sering digunakan sebagai solusi coba untuk persamaan diferensial linear. Meskipun
demikian, bentuk ini memiliki kekurangan. Karena $x(t)$ adalah kuantitas real, maka koefisien
$C_1$ dan $C_2$ harus dipilih sedemikian rupa sehingga $x(t)$ bernilai real. Ada kalanya
kita lebih mudah untuk menggunakan bentuk lain dari Persamaan \eqref{eq:Taylor_5_5}.

Dengan menggunakan persamaan
\begin{equation*}
e^{\pm\mathrm{i}\omega_{0} t} = \cos(\omega_{0} t)\pm\mathrm{i}\sin(\omega_{0} t)
\end{equation*}
%
dapat diperoleh
%
\begin{align*}
x(t) & = (C_{1} + C_{2}) \cos(\omega_{0} t) + \mathrm{i}(C_{1} - C_{2})\sin(\omega_{0} t) \nonumber \\
     & = B_{1}\cos(\omega_{0} t) + B_{2}\sin(\omega_{0} t) \label{eq:Taylor_5-6}
\end{align*}
%
dengan $B_1$ dan $B_2$ dinyatakan dalam koefisien $C_1$ dan $C_2$:
%
\begin{equation}
B_{1} = C_{1} + C_{2} \text{ dan } B_{2} = \mathrm{i}(C_{1} - C_{2})
\label{eq:Taylor_5_7}
\end{equation}
%
Turunan pertama:
\begin{equation*}
\frac{\mathrm{d}x}{\mathrm{d}t}=-B_{1}\omega\sin(\omega t)+B_{2}\omega\cos(\omega t)
\end{equation*}

Misalnya diberikan syarat awal
$x(t=0) = x_{0}$
dan
$\left|\frac{\mathrm{d}x}{\mathrm{d}t}\right|_{t=0} = v_{0} = 0$
maka dapat diperoleh $B_{1} = x_{0}$ dan $B_{2} = 0$.

Untuk syarat awal yang lain:
$x(t=0) = x_0 = 0$
dan
$\left|\frac{\mathrm{d}x}{\mathrm{d}t}\right|_{t=0} = v_{0}$
maka diperole $B_{1} = 0$ dan $B_{2} = \frac{v_{0}}{\omega}$.

\subsection*{Solusi kosinus dengan pergeseran fasa}

Definisikan suatu konstanta lain $A$ sebagai:
\begin{equation}
A = \sqrt{B_{1}^{2} + B_{2}^{2}}
\label{eq:Taylor_5_10}
\end{equation}
Perhatikan bahwa $A$ dapat dianggap sebagai sisi miring segitiga siku-siku dengan
sisi-sisi yang lainnya adalah $B_1$ dan $B_2$, seperti pada Gambar \ref{fig:Taylor_5_4}.

\begin{marginfigure}
\includegraphics[width=\textwidth]{images_priv/Taylor_Fig_5_4.png}
\caption{Konstanta $A$ dan $\delta$ didefinisikan dalam $B_1$ dan $B_2$.}
\label{fig:Taylor_5_4}
\end{marginfigure}

Dengan menggunakan hubungan ini, Persamaan \eqref{eq:Taylor_5_6} dapat
ditulis menjadi:
\begin{align*}
x(t) & = A \left[\frac{B_{1}}{A}\cos(\omega_0 t) + \frac{B_{2}}{A}\sin(\omega_0 t) \right] \\
     & = A\left[ \cos\delta\cos(\omega_0 t)+\sin\delta\sin(\omega_0 t) \right]
\end{align*}
atau
\begin{equation}
  x(t) = A \cos(\omega_0 t - \delta)
  \label{eq:Taylor_5_11}
\end{equation}
dengan $\delta$ adalah pergeseran (sudut) fasa


\subsection*{Bagian real dari eksponensial kompleks}

Koefisien $C_1$ dan $C_2$ dapat dinyatakan dalam $B_1$ dan $B_2$ sebagai
\begin{equation}
C_{1} = \frac{1}{2}(B_{1} - \mathrm{i}B_{2}) \text{ dan }
C_{2} = \frac{1}{2}(B_{1} + \mathrm{i}B_{2})
\label{eq:Taylor_5_12}
\end{equation}
Karena $B_{1}$dan $B_{2}$ bernilai real, hubungan ini menunjukkan bahwa
$C_{1}$ dan $C_{2}$ secara umum bernilai kompleks dan saling konjugat kompleks:
\begin{equation*}
C_{2} = C_{1}^{*}
\end{equation*}
sehingga Persamaan \eqref{eq:Taylor_5_5}
\begin{equation*}
x(t) = C_{1}e^{\mathrm{i}\omega_0 t} + C_{2}e^{-\mathrm{i}\omega_0 t}
\end{equation*}
dapat ditulis menjadi (karena $C_1$ dan $C_2$ adalah konjugat kompleks):
\begin{equation}
x(t) = C_{1}e^{\mathrm{i}\omega_0 t} + C_{1}^{*}e^{-\mathrm{i}\omega_0 t}
\label{eq:Taylor_5_13}
\end{equation}

Sekarang, perhatikan bahwa untuk sembarang bilangan kompleks $z = x + \mathrm{i}y$
kita dapat menghitung
\begin{equation*}
z + z^{*} = (x + \mathrm{i}y) + (x - \mathrm{i}y) = 2x = 2\mathrm{Re}(z)
\end{equation*}
sehingga Persamaan \eqref{eq:Taylor_5_13} dapat ditulis menjadi
\begin{align*}
x(t) & = 2\mathrm{Re}\left(C_{1}e^{\mathrm{i}\omega_0 t} \right) \\
     & = \mathrm{Re}\left(2C_{1}e^{\mathrm{i}\omega_0 t} \right)
\end{align*}
Dengan medefinisikan $C = 2C_{1}$:
\begin{equation*}
x(t) = \mathrm{Re}\left(C e^{\mathrm{\mathrm{i}}\omega_0 t}\right)
\end{equation*}
Karena $C = 2C_1 = B_1 - \mathrm{i}B_2$ (lihat persamaan \eqref{eq:Taylor_5_12})
dan menggunakan Gambar \ref{fig:Taylor_5_4} kita dapat menuliskan
\begin{equation}
C = B_1 - \mathrm{i}B_2 \equiv A e^{-\mathrm{i}\delta}
\end{equation}
dan
\begin{align*}
x(t) & = \mathrm{Re}\left(Ce^{\mathrm{\mathrm{i}}\omega_{0} t}\right) \\
 & = \mathrm{Re}\left(Ae^{\mathrm{-i}\delta}e^{\mathrm{i}\omega_{0} t} \right) \\
 & = \mathrm{Re}\left(Ae^{\mathrm{i}(\omega_{0} t-\delta)}\right)
\end{align*}

Bilangan kompleks $Ae^{\mathrm{i}(\omega_0 t - \delta)}$ bergerak berlawanan arah jarum jam dengan
kecepatan sudut $\omega_0$ di sekitar suatu lingkarang dengan jari-jari $A$. Bagian real dari
bilangan kompleks ini, yaitu $x(t)$ adalah proyeksi dari bilangan kompleks tersebut pada
sumbu real ($x$). Ketika bilangan kompleks ini mengitari lingkaran, proyeksi pada sumbu $x$
berosilasi mundur dan maju pada sumbu-$x$ dengan kecepatan sudut $\omega_0$ dan amplitudo $A$.