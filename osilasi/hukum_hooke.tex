\section{Hukum Hooke}

Tinjau suatu massa yang terhubung dengan ujung sebuah pegas. Hukum Hooke menyatakan
bahwa gaya yang bekerja pada massa apabila massa mengalami simpangan dari titik
setimbangnya adalah:
\begin{equation}
F_{x}(x) = -kx
\end{equation}
di mana $k$ adalah konstanta (pegas) dan $x$ adalah simpangan massa dari keadaan stabil.
Gaya $F_x(x)$ yang bekerja ini dikenal juga sebagai gaya pemulih (\textit{restoring force}).
Perhatikan bahwa gaya ini memiliki arah yang berlawanan dari arah simpangan massa.

Hukum Hooke juga dapat dinyatakan melalui energi potensial $U(x)$ akibat pegas, yang ditulis
\begin{equation}
U(x)=\frac{1}{2}kx^{2}.
\end{equation}
Misalkan sistem berada pada posisi stabil pada $x = x_{0}$, biasanya
diambil sebagai 0, $x_{0} = 0$.
Untuk meninjau perilaku $U(x)$ disekitar posisi setimbang, kita dapat menggunakan ekspansi deret
Taylor dari $U(x)$:
\begin{equation}
U(x) = U(0) + U'(0)x + \frac{1}{2}U''(x)x^{2} + \frac{1}{3!}U'''(x)x^{3} + \cdots
\end{equation}
atau
$$
U(x) = U(0) + \left. \frac{\mathrm{d}U(x)}{\mathrm{d}x}\right |_{x=0} x +
\frac{1}{2} \left. \frac{\mathrm{d}^{2}U(x)}{\mathrm{d}x^{2}}\right |_{x=0} x^{2} +
\frac{1}{3!} \left. \frac{\mathrm{d}^{3}U(x)}{\mathrm{d}x^{3}}\right |_{x=0} x^{3} + \cdots
$$
Dengan asumsi bahwa  $x$ bernilai cukup kecil, kita dapat mengambil tiga suku pertama
dari ekspansi ini:
\begin{itemize}
\item Suku pertama: $U(0)$ bernilai konstan, yang dapat diambil bernilai 0,
\item Suku kedua: $U'(0)$ bernilai 0 karena $x = 0$ adalah titik ekuilibrium
\item Suku ketiga: $U''(0)$ bernilai positif, karena $x = 0$ adalah titik ekuilibrium stabil
\end{itemize}
Dapat diidentifikasi bahwa:
\begin{equation}
k \equiv U''(0) = \left. \frac{\mathrm{d}^{2}U(x)}{\mathrm{d}x^{2}} \right|_{x=0}
\end{equation}

% TODO: Contoh