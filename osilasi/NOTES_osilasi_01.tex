\input{../PREAMBLE_tufte}

\begin{document}


\title{Osilasi Harmonik \\
TF2103}
\author{Fadjar Fathurrahman}
\date{Versi 24 November 2022}
\maketitle

\emph{Materi ini diadaptasi dari Taylor, Classical Mechanics, Bab 5.}

Hampir seluruh sistem yang mengalami pergeseran dari posisi ekuilibriumnya
(stabil) akan mengalami osilasi.
Jika pergeseran yang terjadi cukup kecil, osilasi
bahwa osilasi yang terjadi bersifat harmonik sederhana.

\section{Hukum Hooke}

Tinjau suatu massa yang terhubung dengan ujung sebuah pegas. Hukum Hooke menyatakan
bahwa gaya yang bekerja pada massa apabila massa mengalami simpangan dari titik
setimbangnya adalah:
\begin{equation}
F_{x}(x) = -kx
\end{equation}
di mana $k$ adalah konstanta (pegas) dan $x$ adalah simpangan massa dari keadaan stabil.
Gaya $F_x(x)$ yang bekerja ini dikenal juga sebagai gaya pemulih (\textit{restoring force}).
Perhatikan bahwa gaya ini memiliki arah yang berlawanan dari arah simpangan massa.

Hukum Hooke juga dapat dinyatakan melalui energi potensial $U(x)$ akibat pegas, yang ditulis
\begin{equation}
U(x)=\frac{1}{2}kx^{2}.
\end{equation}
Misalkan sistem berada pada posisi stabil pada $x = x_{0}$, biasanya
diambil sebagai 0, $x_{0} = 0$.
Untuk meninjau perilaku $U(x)$ disekitar posisi setimbang, kita dapat menggunakan ekspansi deret
Taylor dari $U(x)$:
\begin{equation}
U(x) = U(0) + U'(0)x + \frac{1}{2}U''(x)x^{2} + \frac{1}{3!}U'''(x)x^{3} + \cdots
\end{equation}
atau
$$
U(x) = U(0) + \left. \frac{\mathrm{d}U(x)}{\mathrm{d}x}\right |_{x=0} x +
\frac{1}{2} \left. \frac{\mathrm{d}^{2}U(x)}{\mathrm{d}x^{2}}\right |_{x=0} x^{2} +
\frac{1}{3!} \left. \frac{\mathrm{d}^{3}U(x)}{\mathrm{d}x^{3}}\right |_{x=0} x^{3} + \cdots
$$
Dengan asumsi bahwa  $x$ bernilai cukup kecil, kita dapat mengambil tiga suku pertama
dari ekspansi ini:
\begin{itemize}
\item Suku pertama: $U(0)$ bernilai konstan, yang dapat diambil bernilai 0,
\item Suku kedua: $U'(0)$ bernilai 0 karena $x = 0$ adalah titik ekuilibrium
\item Suku ketiga: $U''(0)$ bernilai positif, karena $x = 0$ adalah titik ekuilibrium stabil
\end{itemize}
Dapat diidentifikasi bahwa:
\begin{equation}
k \equiv U''(0) = \left. \frac{\mathrm{d}^{2}U(x)}{\mathrm{d}x^{2}} \right|_{x=0}
\end{equation}

% TODO: Contoh

\section{Gerak harmonik sederhana}

Dengan menggunakan hukum Newton:
\footnote{
Notasi di buku: $\ddot{x}\equiv\dfrac{\mathrm{d}^{2}x}{\mathrm{d}t^{2}}$
}
\begin{equation*}
m \frac{\mathrm{d}^2x}{\mathrm{d}t^2} = F_x
\end{equation*}
%
diperoleh
%
\begin{equation*}
m \frac{\mathrm{d}^2x}{\mathrm{d}t^2}  = -kx
\end{equation*}
%
atau:
%
\begin{equation}
\frac{\mathrm{d}^2x}{\mathrm{d}t^2} = -\omega_0^2 x
\label{eq:Taylor_5_4}
\end{equation}
%
dengan $\omega_0$ adalah frekuensi sudut
%
\begin{equation}
\omega_{0} = \sqrt{\frac{k}{m}}
\end{equation}
%
Ingat juga hubungan frekuensi sudut dengan periode dan frekuensi linear $f$
yaitu $\omega = \dfrac{2 \pi}{T} = 2 \pi f$.

Tugas kita selanjutnya adalah mencari solusi $x(t)$ dari persamaan diferensial
\eqref{eq:Taylor_5_4}. Solusi ini dapat dituliskan dalam berbagai bentuk.



\subsection*{Solusi eksponensial}

Biasanya untuk persamaan diferensial linear, digunakan \textit{ansatz} solusi
coba (\textit{trial solution}) berupa fungsi eksponensial:
\begin{align*}
x_{1}(t) & = e^{\mathrm{i}\omega_0 t} \\
x_{2}(t) & = e^{-\mathrm{i}\omega_0 t}
\end{align*}
%
Dapat dicek bahwa kedua solusi coba tersebut memenuhi persamaan \eqref{eq:Taylor_5_4}.
Misalnya, untuk $x_{1}(t) = e^{\mathrm{i}\omega_0 t}$ diperoleh turunan pertama
%
\begin{equation*}
\frac{\mathrm{d}x_{1}(t)}{\mathrm{d}t} = \mathrm{i}\omega e^{\mathrm{i} \omega_0 t}
\end{equation*}
%
dan turunan kedua
%
\begin{equation*}
\frac{\mathrm{d}^{2}x_{1}(t)}{\mathrm{d}t^{2}} = (\mathrm{i}\omega_0)^{2} e^{\mathrm{i} \omega t} =
-\omega_{0}^{2}e^{\mathrm{i} \omega t}
\end{equation*}
sehingga:
\begin{equation*}
\frac{\mathrm{d}^{2}x_1(t)}{\mathrm{d}t^{2}} = -\omega_0^{2} x_{1}(t)
\end{equation*}
Dengan cara yang sama juga dapat ditunjukkan bahwa $x_{2}(t)$ juga
memenuhi \eqref{eq:Taylor_5_4}.

Solusi umum dari persamaan \eqref{eq:Taylor_5_4} adalah kombinasi linear dari
$x_{1}(t)$ dan $x_{2}(t)$:
\begin{equation}
x(t) = C_{1} e^{\mathrm{i}\omega_{0} t} + C_{2} e^{-\mathrm{i}\omega_{0} t}
\label{eq:Taylor_5_5}
\end{equation}
dengan $C_1$ dan $C_2$ adalah konstanta yang dapat
ditentukan dari syarat awal: $x(t=0)$ dan $\dfrac{\mathrm{d}x}{\mathrm{d}t}(t=0)$.



\subsection*{Solusi sinusoidal (sin dan cos)}

Solusi eksponensial pada Persamaan \eqref{eq:Taylor_5_5} sangat mudah untuk dimanipulasi
sehingga sering digunakan sebagai solusi coba untuk persamaan diferensial linear. Meskipun
demikian, bentuk ini memiliki kekurangan. Karena $x(t)$ adalah kuantitas real, maka koefisien
$C_1$ dan $C_2$ harus dipilih sedemikian rupa sehingga $x(t)$ bernilai real. Ada kalanya
kita lebih mudah untuk menggunakan bentuk lain dari Persamaan \eqref{eq:Taylor_5_5}.

Dengan menggunakan persamaan
\begin{equation*}
e^{\pm\mathrm{i}\omega_{0} t} = \cos(\omega_{0} t)\pm\mathrm{i}\sin(\omega_{0} t)
\end{equation*}
%
dapat diperoleh
%
\begin{align*}
x(t) & = (C_{1} + C_{2}) \cos(\omega_{0} t) + \mathrm{i}(C_{1} - C_{2})\sin(\omega_{0} t) \nonumber \\
     & = B_{1}\cos(\omega_{0} t) + B_{2}\sin(\omega_{0} t) \label{eq:Taylor_5-6}
\end{align*}
%
dengan $B_1$ dan $B_2$ dinyatakan dalam koefisien $C_1$ dan $C_2$:
%
\begin{equation}
B_{1} = C_{1} + C_{2} \text{ dan } B_{2} = \mathrm{i}(C_{1} - C_{2})
\label{eq:Taylor_5_7}
\end{equation}
%
Turunan pertama:
\begin{equation*}
\frac{\mathrm{d}x}{\mathrm{d}t}=-B_{1}\omega\sin(\omega t)+B_{2}\omega\cos(\omega t)
\end{equation*}

Misalnya diberikan syarat awal
$x(t=0) = x_{0}$
dan
$\left|\frac{\mathrm{d}x}{\mathrm{d}t}\right|_{t=0} = v_{0} = 0$
maka dapat diperoleh $B_{1} = x_{0}$ dan $B_{2} = 0$.

Untuk syarat awal yang lain:
$x(t=0) = x_0 = 0$
dan
$\left|\frac{\mathrm{d}x}{\mathrm{d}t}\right|_{t=0} = v_{0}$
maka diperole $B_{1} = 0$ dan $B_{2} = \frac{v_{0}}{\omega}$.

\subsection*{Solusi kosinus dengan pergeseran fasa}

Definisikan suatu konstanta lain $A$ sebagai:
\begin{equation}
A = \sqrt{B_{1}^{2} + B_{2}^{2}}
\label{eq:Taylor_5_10}
\end{equation}
Perhatikan bahwa $A$ dapat dianggap sebagai sisi miring segitiga siku-siku dengan
sisi-sisi yang lainnya adalah $B_1$ dan $B_2$, seperti pada Gambar \ref{fig:Taylor_5_4}.

\begin{marginfigure}
\includegraphics[width=\textwidth]{images_priv/Taylor_Fig_5_4.png}
\caption{Konstanta $A$ dan $\delta$ didefinisikan dalam $B_1$ dan $B_2$.}
\label{fig:Taylor_5_4}
\end{marginfigure}

Dengan menggunakan hubungan ini, Persamaan \eqref{eq:Taylor_5_6} dapat
ditulis menjadi:
\begin{align*}
x(t) & = A \left[\frac{B_{1}}{A}\cos(\omega_0 t) + \frac{B_{2}}{A}\sin(\omega_0 t) \right] \\
     & = A\left[ \cos\delta\cos(\omega_0 t)+\sin\delta\sin(\omega_0 t) \right]
\end{align*}
atau
\begin{equation}
  x(t) = A \cos(\omega_0 t - \delta)
  \label{eq:Taylor_5_11}
\end{equation}
dengan $\delta$ adalah pergeseran (sudut) fasa


\subsection*{Bagian real dari eksponensial kompleks}

Koefisien $C_1$ dan $C_2$ dapat dinyatakan dalam $B_1$ dan $B_2$ sebagai
\begin{equation}
C_{1} = \frac{1}{2}(B_{1} - \mathrm{i}B_{2}) \text{ dan }
C_{2} = \frac{1}{2}(B_{1} + \mathrm{i}B_{2})
\label{eq:Taylor_5_12}
\end{equation}
Karena $B_{1}$dan $B_{2}$ bernilai real, hubungan ini menunjukkan bahwa
$C_{1}$ dan $C_{2}$ secara umum bernilai kompleks dan saling konjugat kompleks:
\begin{equation*}
C_{2} = C_{1}^{*}
\end{equation*}
sehingga Persamaan \eqref{eq:Taylor_5_5}
\begin{equation*}
x(t) = C_{1}e^{\mathrm{i}\omega_0 t} + C_{2}e^{-\mathrm{i}\omega_0 t}
\end{equation*}
dapat ditulis menjadi (karena $C_1$ dan $C_2$ adalah konjugat kompleks):
\begin{equation}
x(t) = C_{1}e^{\mathrm{i}\omega_0 t} + C_{1}^{*}e^{-\mathrm{i}\omega_0 t}
\label{eq:Taylor_5_13}
\end{equation}

Sekarang, perhatikan bahwa untuk sembarang bilangan kompleks $z = x + \mathrm{i}y$
kita dapat menghitung
\begin{equation*}
z + z^{*} = (x + \mathrm{i}y) + (x - \mathrm{i}y) = 2x = 2\mathrm{Re}(z)
\end{equation*}
sehingga Persamaan \eqref{eq:Taylor_5_13} dapat ditulis menjadi
\begin{align*}
x(t) & = 2\mathrm{Re}\left(C_{1}e^{\mathrm{i}\omega_0 t} \right) \\
     & = \mathrm{Re}\left(2C_{1}e^{\mathrm{i}\omega_0 t} \right)
\end{align*}
Dengan medefinisikan $C = 2C_{1}$:
\begin{equation*}
x(t) = \mathrm{Re}\left(C e^{\mathrm{\mathrm{i}}\omega_0 t}\right)
\end{equation*}
Karena $C = 2C_1 = B_1 - \mathrm{i}B_2$ (lihat persamaan \eqref{eq:Taylor_5_12})
dan menggunakan Gambar \ref{fig:Taylor_5_4} kita dapat menuliskan
\begin{equation}
C = B_1 - \mathrm{i}B_2 \equiv A e^{-\mathrm{i}\delta}
\end{equation}
dan
\begin{align*}
x(t) & = \mathrm{Re}\left(Ce^{\mathrm{\mathrm{i}}\omega_{0} t}\right) \\
 & = \mathrm{Re}\left(Ae^{\mathrm{-i}\delta}e^{\mathrm{i}\omega_{0} t} \right) \\
 & = \mathrm{Re}\left(Ae^{\mathrm{i}(\omega_{0} t-\delta)}\right)
\end{align*}

Bilangan kompleks $Ae^{\mathrm{i}(\omega_0 t - \delta)}$ bergerak berlawanan arah jarum jam dengan
kecepatan sudut $\omega_0$ di sekitar suatu lingkarang dengan jari-jari $A$. Bagian real dari
bilangan kompleks ini, yaitu $x(t)$ adalah proyeksi dari bilangan kompleks tersebut pada
sumbu real ($x$). Ketika bilangan kompleks ini mengitari lingkaran, proyeksi pada sumbu $x$
berosilasi mundur dan maju pada sumbu-$x$ dengan kecepatan sudut $\omega_0$ dan amplitudo $A$.


\section{Osilasi teredam}

Misalkan sekarang selain dari gaya pemulih juga
bekerja gaya gesek atau gaya redam, yang disebabkan misalnya oleh
gesekan dengan udara atau fluida lain.
Gaya redam ini biasanya dimodelkan sebagai
\begin{equation}
F_{\mathrm{damp}} = -bv = -b\frac{\mathrm{d}x}{\mathrm{d}t}
\end{equation}
dengan $b$ adalah suatu konstanta.
Pada kasus ini, persamaan gerak dari osilator menjadi sebagai berikut:
\begin{equation}
m\frac{\mathrm{d}^{2}x}{\mathrm{d}t^{2}}+b\frac{\mathrm{d}x}{\mathrm{d}t}+kx=0
\end{equation}
Dengan mendefinisikan konstanta $\beta$, disebut konstanta redaman, \emph{damping constant}
sebagai berikut
\begin{equation}
\beta = \frac{b}{2m}
\end{equation}
dan frekuensi alami (\emph{natural frequency})
\begin{equation}
\omega_{0}=\sqrt{\frac{k}{m}},
\end{equation}
persamaan osilator teredam dapat ditulis menjadi
\begin{equation}
\frac{\mathrm{d}^{2}x}{\mathrm{d}t^{2}} + 2\beta \frac{\mathrm{d}x}{\mathrm{d}t} + \omega_{0}^{2}x = 0
\label{eq:Taylor_5_28}
\end{equation}

Persamaan ini masih merupakan persamaan diferensial orde dua, linear,
dan homogen. Jika kita dapat menemukan dua solusi independen $x_{1}(t)$
dan $x_{2}(t)$ pada persamaan ini, maka solusi umum dari persamaan
ini dapat dituliskan sebagai superposisi dari dua solusi, yaitu $C_{1}x_{1}(t)+C_{2}x_{2}(t)$.

Seperti kasus osilator harmonik tanpa redaman,
\textit{ansatz} solusi coba yang sering digunakan untuk persamaan diferensial ini adalah
\begin{equation*}
x(t) = e^{rt}
\end{equation*}
Dapat dihitung
\begin{equation*}
\frac{\mathrm{d}x}{\mathrm{d}t} = re^{rt}
\end{equation*}
dan
\begin{equation*}
\frac{\mathrm{d}^{2}x}{\mathrm{d}t^{2}} = r^{2}e^{rt}
\end{equation*}
Dengan melakukan substitusi pada persamaan diferensial untuk osilasi
teredam diperoleh persamaan bantu (\textit{auxiliary equation}) dalam $r$:
\begin{equation}
r^{2} + 2\beta r + \omega_{0}^{2} = 0
\end{equation}
Persamaan ini adalah persamaan kuadrat dengan akar-akar sebagai berikut:
\begin{align*}
r_{1} & =\frac{-2\beta + \sqrt{(2\beta)^{2} - 4(1)(\omega_{0}^{2})}}{2(1)} = -\beta+\sqrt{\beta^{2}-\omega_{0}^{2}}\\
r_{2} & =\frac{-2\beta - \sqrt{(2\beta)^{2} - 4(1)(\omega_{0}^{2})}}{2(1)} = -\beta-\sqrt{\beta^{2}-\omega_{0}^{2}}
\end{align*}
Maka dengan menyelesaikan persamaan bantu, kita mendapatkan dua fungsi $e^{r_{1}t}$ dan $e^{r_{2}t}$
yang merupakan dua solusi independen. 
Solusi umumnya dari Persamaan \eqref{eq:Taylor_5_28} adalah:
\begin{align*}
x(t) & = C_{1}e^{r_{1}t}+C_{2}e^{r_{2}t} \\
 & = e^{-\beta t} \left( C_1 e^{\sqrt{\beta^2 - \omega_0} t} + C_2 e^{ -\sqrt{\beta^2 - \omega_0} t} \right)
\end{align*}

Sekarang, bergantung dari nilai-nilai $r_1$ dan $r_2$ (atau parameter $\beta$ dan $\omega_0$),
kita akan meninjau beberapa kasus yang mungkin.


\subsection*{Osilasi tidak teredam}

Kasus pertama yang akan ditinjau adalah ketika $\beta = 0$, atau tidak ada redaman sama sekali.
Kasus pada kasus ini $r_1$ dan $r_2$ akan bernilai imajiner murni $r_{1,2} = \pm \omega_0$ yang
sudah dibahas pada bagian sebelumnya.

Untuk selanjutnya kita akan mengasumsikan nilai $\beta > 0$.

\subsection*{Osilasi teredam lemah (underdamped)}

Pada kasus $\beta > 0$ dan $\beta < \omega_{0}$ akar-akar $r_{1}$ dan $r_{2}$ akan
bernilai kompleks karena kuantitas $\beta**2 - \omega_0^2$ akan bernilai negatif.
Kita dapat menuliskan
\begin{equation*}
\sqrt{\beta^{2}  -\omega_{0}^{2}} = \mathrm{i} \sqrt{\omega_{0}^{2}-\beta^{2}} = \mathrm{i}\omega_{1}
\end{equation*}
dengan 
\begin{equation*}
\omega_{1} = \sqrt{\omega_{0}^{2} - \beta^{2}}
\end{equation*}
Kasus ini juga sering disebut sebagai kasus \textit{underdamped}.

Solusi $x(t)$ dapat ditulis sebagai berikut
\begin{equation}
x(t) = e^{-\beta t} \left( C_{1}e^{\mathrm{i}\omega_{1}t} + C_{2}e^{-\mathrm{i}\omega_{1}t} \right)
\end{equation}
Karena bentuk eksponensial di dalam tanda kurung dapat diubah menjadi fungsi cosinus
dengan pergeseran fasa, maka juga dapat ditulis
\begin{equation}
x(t) = Ae^{-\beta t}\cos(\omega_{1}t - \delta)
\end{equation}


\subsection*{Osilasi teredam kuat (overdamped)}

Tinjau kasus di mana $\beta > \omega_{0}$ sehingga akar-akar $r_{1}$ dan $r_{2}$ bernilai real.

Solusi $x(t)$ dapat ditulis menjadi:
\begin{equation}
x(t) = C_{1}e^{ - \left( \beta - \sqrt{\beta^{2}-\omega_{0}^{2}} \right) t} +
       C_{2}e^{ - \left( \beta + \sqrt{\beta^{2}-\omega_{0}^{2}} \right) t}
\end{equation}
Kasus ini juga sering disebut sebagai \textit{overdamped}.
     
Perhatikan bahwa kedua pangkat dari eksponensial tersebut adalah negatif, sehingga $x(t)$ akan menuju 0 ketika
$t \rightarrow \infty$.
Perhatikan juga bahwa suku pertama akan meluruh lebih lama daripada suku kedua karena nilai
eksponen negatif suku pertama lebih kecil daripada suku
kedua. Dalam jangka panjang suku ini akan mendominasi, karena suku kedua sudah meluruh lebih dahulu (lebih cepat
mendekati nol). Oleh karena itu laju di mana gerakan akan berhenti dapat dikarakterisasi oleh
koefisien dari ekponensial pertama yaitu
\begin{equation}
\text{parameter peluruhan} = \beta - \sqrt{\beta^2 - \omega_0^2}
\end{equation}
Perhatikan bahwa, laju peluruhan akan menjadi lebih kecil apabila konstanta $\beta$
menjadi semakin besar.


\subsection*{Osilasi teredam kritis (critically damped)}

Pada kasus ini $\beta=\omega_{0}$ sehingga
akar-akar $r_{1}$ dan $r_{2}$ bernilai sama, yaitu $r_{1} = r_{2} = \beta$.

Karena hanya ada satu solusi khusus, kita perlu mencari satu solusi khusus
lagi agar dapat menuliskan solusi umum. Solusi khusus tambahan yang
dapat digunakan adalah
\begin{equation}
x(t) = te^{-\beta t}
\end{equation}
sehingga solusi umum yang diperoleh adalah
\begin{align*}
x(t) & = C_{1}e^{-\beta t} + C_{2}te^{-\beta t} \\
     & = e^{-\beta t}\left(C_{1} + C_{2}t \right)
\end{align*}


\section{Osilasi terpaksa}

Misalkan gaya pemaksa luar (\textit{external driving force}) yang bekerja adalah
$F(t)$, dan juga bekerja gaya redaman $-bv$, maka gaya total yang
bekerja pada osilator adalah $-bv-kx+F(t)$, sehingga diperoleh persamaan
gerak sebagai berikut:
\[
m\frac{\mathrm{d}^{2}x}{\mathrm{d}t^{2}}+b\frac{\mathrm{d}x}{\mathrm{d}t}+kx=0
\]
dengan mendefinisikan $\beta=b/(2m),$$\omega_{0}^{2}=k/m$, dan
\[
f(t)=\frac{F(t)}{m}
\]
adalah gaya persatuan massa. Dengan notasi ini dapat dituliskan persamaan
gerak untuk osilator sebagai berikut:
\[
\frac{\mathrm{d}^{2}x}{\mathrm{d}t^{2}}+2\beta\frac{\mathrm{d}x}{\mathrm{d}t}+\omega_{0}^{2}x=f(t)
\]

\subsection{Operator diferensial linear}

\[
\mathbb{D}=\frac{\mathrm{d}^{2}}{\mathrm{d}t^{2}}+2\beta\frac{\mathrm{d}}{\mathrm{d}t}+\omega_{0}^{2}
\]

dengan menggunakan notasi ini, persamaan diferensial untuk osilator
harmonik terpaksa dapat ditulis menjadi
\[
\mathbb{D}x=f
\]

dengan $x$ dan $f$ adalah fungsi dari $t$. Perhatikan bahwa $\mathbb{D}$
adalah operator linear, karena:
\[
\mathbb{D}(ax)=a\mathbb{D}x
\]

untuk suatu skalar $a$ dan
\[
\mathbb{D}(x_{1} + x_{2}) = \mathbb{D}x_{1} + \mathbb{D}x_{2}
\]
dengan $x_{1}$ dan $x_{2}$ adalah sembarang fungsi dari $t$.

Dua kondisi di atas dapat digabung menjadi:
\[
\mathbb{D}(ax_{1}+bx_{2})=a\mathbb{D}x_{1}+b\mathbb{D}x_{2}
\]
untuk sembarang konstanta $a$ dan $b$ dan dua fungsi $x_{1}(t)$
dan $x_{2}(t)$. Sembarang operator yang memenuhi syarat ini disebut
operator linear.

Dengan menggunakan notasi ini, persamaan diferensial untuk osilator
harmonik tanpa gaya paksa atau gaya eksternal adalah:
\[
\mathbb{D}x = 0
\]
Persamaan ini juga sering disebut persamaan homogen, karena untuk
setiap suku yang terlibat hanya $x$ atau salah satu dari turunannya
tepat sekali, sedangkan persamaan $\mathbb{D}x=f$ disebut juga dengan
persamaan inhomogen, karena mengandung suku $f$ yang tidak melibatkan
$x$ sama sekali.

Misalkan kita telah mendapatkan suatu solusi, $x_{p}(t)$ yang memenuhi
\begin{equation}
\mathbb{D}x_{p} = f
\label{eq:Taylor_5_53}
\end{equation}
Solusi $x_{p}(t)$ ini disebut sebagai solusi khusus (particular solution).

Tinjau persamaan homogen $\mathbb{D}x=0$ dan misalkan kita telah
mendapatkan solusi $x_{h}(t)$ yang memenuhi
\[
\mathbb{D}x_{h}=0
\]
Solusi ini disebut sebagai solusi homogen. Untuk persamaan diferensial
osilator harmonik:
\[
x_{h}(t)=C_{1}e^{r_{1}t}+C_{2}e^{r_{2}t}
\]
di mana kedua suku eksponensial ini menuju nol untuk $t\rightarrow\infty$
(dengan syarat $r_{1}$ dan $r_{2}$ tidak bernilai real positif).

Jika $x_{p}$ adalah solusi khusus dari \ref{eq:Taylor_5_53}, maka
$x_{p}+x_{h}$ adalah juga adalah satu solusi yang lain, karena juga
memenuhi persamaan diferensial $\mathbb{D}x=f$:
\begin{align*}
\mathbb{D}(x_{p}+x_{h}) & =\mathbb{D}x_{p}+\mathbb{D}x_{h}\\
 & =f + 0 = f
\end{align*}


\subsection{Kasus khusus: gaya pemaksa cosinus}

Tinjau kasus khusus di mana
\begin{equation}
f(t) = f_{0}\cos(\omega t)
\end{equation}
dengan $f_{0}$ menyatakan amplitudo dari gaya pemaksa persatuan massa
dan $\omega$ adalah frekuensi sudut dari gaya pemaksa (\textit{driving frequency}).
Perhatikan bahwa $\omega$ berbeda atau independen dari $\omega_0$

Persamaan gerak menjadi 
\[
\frac{\mathrm{d}^{2}x}{\mathrm{d}t^{2}}+2\beta\frac{\mathrm{d}x}{\mathrm{d}t}+\omega_{0}^{2}x=f_{0}\cos(\omega t)
\]

Untuk sembarang solusi dari persamaan ini, terdapat suatu solusi dari
persamaan yang sama, namun dengan suku cos pada RHS diganti dengan
fungsi sin, karena cos dan sin hanya dibedakan dengan sudut fasa.
Dengan kata lain, terdapat suatu fungsi $y(t)$ yang memenuhi
\[
\frac{\mathrm{d}^{2}y}{\mathrm{d}t^{2}}+2\beta\frac{\mathrm{d}y}{\mathrm{d}t}+\omega_{0}^{2}y=f_{0}\sin(\omega t)
\]

Sekarang, definisikan fungsi kompleks:
\[
z(t)=x(t)+\mathrm{i}y(t)
\]

Jika kita mengalikan persamaan sin dengan $\mathrm{i}$ dan menjumlahkannya
ke persamaan dengan cos, diperoleh
\[
\frac{\mathrm{d}^{2}z}{\mathrm{d}t^{2}}+2\beta\frac{\mathrm{d}z}{\mathrm{d}t}+\omega_{0}^{2}z=f_{0}e^{\mathrm{i}\omega t}
\]
Persamaan ini lebih mudah untuk diselesaikan.

Gunakan bentuk solusi
\[
z(t)=Ce^{\mathrm{i}\omega t}
\]
yang memberikan
\[
(-\omega^{2}+2\mathrm{i}\beta\omega+\omega_{0}^{2})Ce^{\mathrm{i}\omega t}=f_{0}e^{\mathrm{i}\omega t}
\]
Agar dapat menjadi solusi, kita memerlukan
\[
C=\frac{f_{0}}{\omega_{0}^{2}-\omega^{2}+2\mathrm{i}\beta\omega}
\]
Tulis koefisien $C$ (yang merupakan bilangan kompleks) dalam bentuk
\[
C=Ae^{-\mathrm{i}\delta}
\]
di mana $A$ dan $\delta$ adalah bilangan real.

Nilai $A$ dapat diperoleh dari
\begin{align*}
A^{2} & =CC^{*}=\frac{f_{0}}{\omega_{0}^{2}-\omega^{2}+2\mathrm{i}\beta\omega}\cdot\frac{f_{0}}{\omega_{0}^{2}-\omega^{2}-2\mathrm{i}\beta\omega}\\
 & =\frac{f_{0}^{2}}{\left(\omega_{0}^{2}-\omega^{2}\right)^{2}+4\beta^{2}\omega^{2}}
\end{align*}

Amplitudo $A$ bernilai paling besar ketika $\omega_{0}\approx\omega$,
sehingga penyebut bernilai paling kecil. Dengan kata lain, osilator
akan merespon paling baik ketika diberikan gaya periodik dengan
frekuensi $\omega$ yang dekat dengan frekuensi natural dari osilator.



\bibliographystyle{chicago}
\bibliography{BIBLIO}

\end{document}
