\input{../PREAMBLE_tufte}

\begin{document}


\title{Osilasi Harmonik \\
TF2103}
\author{Fadjar Fathurrahman}
\date{Versi 24 November 2022}
\maketitle

Materi ini diadaptasi dari Taylor, Classical Mechanics, Bab 5.

Hampir seluruh sistem yang mengalami pergeseran dari posisi ekuilibriumnya
(stabil) akan mengalami osilasi.
Jika pergeseran yang terjadi cukup kecil, osilasi
bahwa osilasi yang terjadi bersifat harmonik sederhana.

\section{Hukum Hooke}

Tinjau suatu massa yang terhubung dengan ujung sebuah pegas. Hukum Hooke menyatakan
bahwa gaya yang bekerja pada massa apabila massa mengalami simpangan dari titik
setimbangnya adalah:
\begin{equation}
F_{x}(x) = -kx
\end{equation}
di mana $k$ adalah konstanta (pegas) dan $x$ adalah simpangan massa dari keadaan stabil.
Gaya $F_x(x)$ yang bekerja ini dikenal juga sebagai gaya pemulih (\textit{restoring force}).
Perhatikan bahwa gaya ini memiliki arah yang berlawanan dari arah simpangan massa.

Hukum Hooke juga dapat dinyatakan melalui energi potensial $U(x)$ akibat pegas, yang ditulis
\begin{equation}
U(x)=\frac{1}{2}kx^{2}.
\end{equation}
Misalkan sistem berada pada posisi stabil pada $x = x_{0}$, biasanya
diambil sebagai 0, $x_{0} = 0$.
Untuk meninjau perilaku $U(x)$ disekitar posisi setimbang, kita dapat menggunakan ekspansi deret
Taylor dari $U(x)$:
\begin{equation}
U(x) = U(0) + U'(0)x + \frac{1}{2}U''(x)x^{2} + \frac{1}{3!}U'''(x)x^{3} + \cdots
\end{equation}
atau
$$
U(x) = U(0) + \left. \frac{\mathrm{d}U(x)}{\mathrm{d}x}\right |_{x=0} x +
\frac{1}{2} \left. \frac{\mathrm{d}^{2}U(x)}{\mathrm{d}x^{2}}\right |_{x=0} x^{2} +
\frac{1}{3!} \left. \frac{\mathrm{d}^{3}U(x)}{\mathrm{d}x^{3}}\right |_{x=0} x^{3} + \cdots
$$
Dengan asumsi bahwa  $x$ bernilai cukup kecil, kita dapat mengambil tiga suku pertama
dari ekspansi ini:
\begin{itemize}
\item Suku pertama: $U(0)$ bernilai konstan, yang dapat diambil bernilai 0,
\item Suku kedua: $U'(0)$ bernilai 0 karena $x = 0$ adalah titik ekuilibrium
\item Suku ketiga: $U''(0)$ bernilai positif, karena $x = 0$ adalah titik ekuilibrium stabil
\end{itemize}
Dapat diidentifikasi bahwa:
\begin{equation}
k \equiv U''(0) = \left. \frac{\mathrm{d}^{2}U(x)}{\mathrm{d}x^{2}} \right|_{x=0}
\end{equation}

% TODO: Contoh

\section{Gerak harmonik sederhana}

Dengan menggunakan hukum Newton:
\footnote{
Notasi di buku: $\ddot{x}\equiv\dfrac{\mathrm{d}^{2}x}{\mathrm{d}t^{2}}$
}
\begin{equation*}
m \frac{\mathrm{d}^2x}{\mathrm{d}t^2} = F_x
\end{equation*}
diperoleh
\begin{equation*}
m \frac{\mathrm{d}^2x}{\mathrm{d}t^2}  = -kx
\end{equation*}
atau:
\begin{equation}
\frac{\mathrm{d}^2x}{\mathrm{d}t^2} = -\omega_0^2 x
\label{eq:Taylor_5_4}
\end{equation}
dengan $\omega_0$ adalah frekuensi sudut
\begin{equation}
\omega_{0} = \sqrt{\frac{k}{m}}
\end{equation}
Ingat juga hubungan frekuensi sudut dengan periode dan frekuensi linear $f$
yaitu $\omega = \dfrac{2 \pi}{T} = 2 \pi f$.

Tugas kita selanjutnya adalah mencari solusi $x(t)$ dari persamaan diferensial
\eqref{eq:Taylor_5_4}.

\subsection*{Solusi eksponensial}

\[
x_{1}(t)=e^{\mathrm{i}\omega t}
\]
\[
x_{2}(t)=e^{-\mathrm{i}\omega t}
\]

Cek:
\[
\frac{\mathrm{d}^{2}x(t)}{\mathrm{d}t^{2}}=-\omega^{2}x(t)
\]

Untuk $x(t)=e^{\mathrm{i}\omega t}$:
\[
\frac{\mathrm{d}x(t)}{\mathrm{d}t}=\mathrm{i}\omega e^{\mathrm{i\omega t}}
\]
\[
\frac{\mathrm{d}^{2}x(t)}{\mathrm{d}t^{2}}=(\mathrm{i}\omega)^{2}e^{\mathrm{i\omega t}}=-\omega^{2}e^{\mathrm{i\omega t}}
\]

Solusi umum:
\begin{align*}
x(t) & =C_{1}x_{1}(t)+C_{2}x_{2}(t)\\
 & =C_{1}e^{\mathrm{i}\omega t}+C_{2}e^{-\mathrm{i}\omega t}
\end{align*}

$C_{1}$dan $C_{2}$ ditentukan dari syarat awal: $x(t=0)$ dan $\dfrac{\mathrm{d}x}{\mathrm{d}t}(t=0)$

\subsection*{Solusi sinusoidal (sin dan cos)}

\[
e^{\pm\mathrm{i}\omega t}=\cos(\omega t)\pm\mathrm{i}\sin(\omega t)
\]

Substitusi ke solusi umum:
\begin{align*}
x(t) & =(C_{1}+C_{2})\cos(\omega t)+\mathrm{i}(C_{1}-C_{2})\sin(\omega t)\\
 & =B_{1}\cos(\omega t)+B_{2}\sin(\omega t)
\end{align*}

$B_{1}=C_{1}+C_{2}$ dan $B_{2}=\mathrm{i}(C_{1}-C_{2})$

Turunan pertama:
\[
\frac{\mathrm{d}x}{\mathrm{d}t}=-B_{1}\omega\sin(\omega t)+B_{2}\omega\cos(\omega t)
\]

Contoh syarat awal (1):
\[
x(t=0)=x_{0}
\]

\[
\left|\frac{\mathrm{d}x}{\mathrm{d}t}\right|_{t=0}=v_{0}=0
\]

maka $B_{1}=x_{0}$ dan $B_{2}=0$.

Contoh syarat awal yang lain (2):
\[
x(t=0)=0
\]

\[
\left|\frac{\mathrm{d}x}{\mathrm{d}t}\right|_{t=0}=v_{0}
\]

maka $B_{1}=0$ dan $B_{2}=\dfrac{v_{0}}{\omega}$.

\subsection*{Phase-shifted cosine solution}

Definisikan konstanta lain:
\[
A=\sqrt{B_{1}^{2}+B_{2}^{2}}
\]

$A$: sisi miring segitiga

Tuliskan kembali solusi
\begin{align*}
x(t) & =A\left[\frac{B_{1}}{A}\cos(\omega t)+\frac{B_{2}}{A}\sin(\omega t)\right]\\
 & =A\left[\cos\delta\cos(\omega t)+\sin\delta\sin(\omega t)\right]\\
 & =A\cos(\omega t-\delta)
\end{align*}

$\delta:$ phase shift

\subsection*{Real part of complex exponential}

\[
C_{1}=\frac{1}{2}(B_{1}-\mathrm{i}B_{2})
\]

\[
C_{2}=\frac{1}{2}(B_{1}+\mathrm{i}B_{2})
\]

$B_{1}$dan $B_{2}$ bernilai real, $C_{1}$dan $C_{2}$secara umum
bernilai kompleks, dan saling konjugat kompleks.
\[
C_{2}=C_{1}^{*}
\]

sehingga
\[
x(t)=C_{1}e^{\mathrm{i}\omega t}+C_{2}e^{-\mathrm{i}\omega t}
\]
\[
x(t)=C_{1}e^{\mathrm{i}\omega t}+C_{1}^{*}e^{-\mathrm{i}\omega t}
\]

Untuk sembarang bilangan kompleks $z=x+\mathrm{i}y$
\[
z+z^{*}=(x+\mathrm{i}y)+(x-\mathrm{i}y)=2x=2\mathrm{Re}(z)
\]

Sehingga:
\[
x(t)=C_{1}e^{\mathrm{i}\omega t}+C_{1}^{*}e^{-\mathrm{i}\omega t}
\]

dapat ditulis menjadi
\[
x(t)=2\mathrm{Re}\left(C_{1}e^{\mathrm{i}\omega t}\right)
\]
\[
x(t)=\mathrm{Re}\left(2C_{1}e^{\mathrm{i}\omega t}\right)
\]

Definisikan $C=2C_{1}:$
\[
x(t)=\mathrm{Re}\left(Ce^{\mathrm{\mathrm{i}}\omega t}\right)
\]

\[
C=B_{1}-\mathrm{i}B_{2}=Ae^{\mathrm{-i}\delta}=A\left(\cos\delta-\mathrm{i}\sin\delta\right)
\]

Substitusi $C$:
\begin{align*}
x(t) & =\mathrm{Re}\left(Ce^{\mathrm{\mathrm{i}}\omega t}\right)\\
 & =\mathrm{Re}\left(Ae^{\mathrm{-i}\delta}e^{\mathrm{i}\omega t}\right)\\
 & =\mathrm{Re}\left(Ae^{\mathrm{i}(\omega t-\delta)}\right)
\end{align*}

\bibliographystyle{chicago}
\bibliography{BIBLIO}

\end{document}
