\section{Osilasi harmonik terpaksa}

Misalkan gaya pemaksa luar (\textit{external driving force}) yang bekerja adalah
$F(t)$, dan juga bekerja gaya redaman $-bv$, maka gaya total yang
bekerja pada osilator adalah $-bv-kx+F(t)$, sehingga diperoleh persamaan
gerak sebagai berikut:
\[
m\frac{\mathrm{d}^{2}x}{\mathrm{d}t^{2}}+b\frac{\mathrm{d}x}{\mathrm{d}t}+kx=0
\]
dengan mendefinisikan $\beta=b/(2m),$$\omega_{0}^{2}=k/m$, dan
\[
f(t)=\frac{F(t)}{m}
\]
adalah gaya persatuan massa. Dengan notasi ini dapat dituliskan persamaan
gerak untuk osilator sebagai berikut:
\[
\frac{\mathrm{d}^{2}x}{\mathrm{d}t^{2}}+2\beta\frac{\mathrm{d}x}{\mathrm{d}t}+\omega_{0}^{2}x=f(t)
\]

\subsection{Operator diferensial linear}

Tinjau operator diferensial $\mathbb{D}$ yang didefinisikan sebagai
\begin{equation}
\mathbb{D}=\frac{\mathrm{d}^{2}}{\mathrm{d}t^{2}}+2\beta\frac{\mathrm{d}}{\mathrm{d}t}+\omega_{0}^{2}
\end{equation}
Dengan menggunakan notasi ini, persamaan diferensial untuk osilator
harmonik terpaksa dapat ditulis menjadi
\begin{equation*}
\mathbb{D}x=f
\end{equation*}
dengan $x$ dan $f$ adalah fungsi dari $t$.

Perhatikan bahwa untuk operator $\mathbb{D}$
berlaku hubungan berikut.
\begin{equation*}
\mathbb{D}(ax) = a\mathbb{D}x
\end{equation*}
untuk suatu skalar $a$ dan
\begin{equation*}
\mathbb{D}(x_{1} + x_{2}) = \mathbb{D}x_{1} + \mathbb{D}x_{2}
\end{equation*}
dengan $x_{1}$ dan $x_{2}$ adalah sembarang fungsi dari $t$.
Dua kondisi di atas dapat digabung menjadi:
\begin{equation*}
\mathbb{D}(ax_{1} + bx_{2})=a\mathbb{D}x_{1}+b\mathbb{D}x_{2}
\end{equation*}
untuk sembarang konstanta $a$ dan $b$ dan dua fungsi $x_{1}(t)$
dan $x_{2}(t)$. Operator yang memenuhi syarat ini disebut
operator linear.

Dengan menggunakan notasi ini, persamaan diferensial untuk osilator
harmonik tanpa gaya paksa atau gaya eksternal adalah:
\begin{equation}
\mathbb{D}x = 0
\label{eq:Taylor_5_52}
\end{equation}
Untuk operator linear berlaku prinsip superposisi. Jika $x_1$ dan $x_2$ adalah solusi
dari Persamaan \eqref{eq:Taylor_5_52}, maka $ax_1 + bx_2$ juga merupakan solusi. Kita
sudah menggunakan prinsip ini sebelumnya dalam menyelesaikan persamaan harmonik
sederhana.

Persamaan \eqref{eq:Taylor_5_52} juga sering disebut persamaan homogen, karena untuk
setiap suku yang terlibat hanya $x$ atau salah satu dari turunannya
tepat sekali, sedangkan persamaan $\mathbb{D}x = f$ disebut juga dengan
persamaan inhomogen, karena mengandung suku $f$ yang tidak melibatkan
$x$ sama sekali. Tugas kita sekarang adalah menyelesaikan persamaan inhomogen ini.
Untungnya, tugas ini menjadi cukup mudah karena kita telah mendapatkan
solusi untuk persamaan homogen.

Misalkan kita telah mendapatkan suatu solusi, $x_{p}(t)$ yang memenuhi
\begin{equation}
\mathbb{D}x_{p} = f
\label{eq:Taylor_5_53}
\end{equation}
Solusi $x_{p}(t)$ ini disebut sebagai solusi khusus (\textit{particular solution}).
Tinjau persamaan homogen $\mathbb{D}x=0$ dan misalkan kita telah
mendapatkan solusi $x_{h}(t)$ yang memenuhi
\begin{equation*}
\mathbb{D}x_{h}=0
\end{equation*}
Solusi ini disebut sebagai solusi homogen. Kita telah mendapatkan solusi ini
yaitu:
\[
x_{h}(t) = C_{1}e^{r_{1}t} + C_{2}e^{r_{2}t}
\]
di mana kedua suku eksponensial ini menuju nol untuk $t\rightarrow\infty$
(dengan syarat $r_{1}$ dan $r_{2}$ tidak bernilai real positif).

Jika $x_{p}$ adalah solusi khusus dari \ref{eq:Taylor_5_53}, maka
$x_{p}+x_{h}$ adalah juga adalah satu solusi yang lain, karena juga
memenuhi persamaan diferensial $\mathbb{D}x=f$:
\begin{align*}
\mathbb{D}(x_{p}+x_{h}) & =\mathbb{D}x_{p}+\mathbb{D}x_{h}\\
 & =f + 0 = f
\end{align*}
Artinya, yang kita perlukan adalah tinggal mencari satu solusi khusus $x_{p}(t)$
dan kita telah mendapatkan solusi umum dari persamaan inhomogen.

\subsection{Kasus khusus: gaya pemaksa cosinus}

Dalam aplikasi, ada beberapa bentuk $f(t)$ yang sering digunakan.
Kita meninjau satu kasus khusus, yaitu
\begin{equation}
f(t) = f_{0}\cos(\omega t)
\end{equation}
dengan $f_{0}$ menyatakan amplitudo dari gaya pemaksa persatuan massa
dan $\omega$ adalah frekuensi sudut dari gaya pemaksa (\textit{driving frequency}).
Perhatikan bahwa $\omega$ berbeda atau independen dari $\omega_0$

Persamaan diferensial yang kita tinjau menjadi
\begin{equation}
\frac{\mathrm{d}^{2}x}{\mathrm{d}t^{2}} + 
2\beta\frac{\mathrm{d}x}{\mathrm{d}t} + \omega_{0}^{2}x = f_{0}\cos(\omega t)
\end{equation}
Kita akan menggunakan suatu trik agar persamaan diferensial ini menjadi lebih
mudah untuk diselesaikan, yaitu dengan menyelesaikan persamaan diferensial
lain yang masih terkait.
Untuk sembarang solusi dari persamaan ini, terdapat suatu solusi dari
persamaan yang sama, namun dengan suku cos pada RHS diganti dengan
fungsi sin, karena cos dan sin hanya dibedakan dengan sudut fasa.
Dengan kata lain, terdapat suatu fungsi $y(t)$ yang memenuhi
\begin{equation*}
\frac{\mathrm{d}^{2}y}{\mathrm{d}t^{2}} +
2\beta\frac{\mathrm{d}y}{\mathrm{d}t} + \omega_{0}^{2}y = f_{0}\sin(\omega t)
\end{equation*}

Sekarang, definisikan fungsi kompleks:
\begin{equation*}
z(t) = x(t) + \mathrm{i}y(t)
\end{equation*}
Jika kita mengalikan persamaan sin dengan $\mathrm{i}$ dan menjumlahkannya
ke persamaan dengan cos, diperoleh
\begin{equation*}
\frac{\mathrm{d}^{2}z}{\mathrm{d}t^{2}} +
2\beta\frac{\mathrm{d}z}{\mathrm{d}t} + \omega_{0}^{2}z = f_{0}e^{\mathrm{i}\omega t}
\end{equation*}
Persamaan ini lebih mudah untuk diselesaikan.

Gunakan bentuk solusi
\begin{equation*}
z(t) = Ce^{\mathrm{i}\omega t}
\end{equation*}
yang memberikan
\begin{equation*}
(-\omega^{2}+2\mathrm{i}\beta\omega+\omega_{0}^{2})Ce^{\mathrm{i}\omega t}=f_{0}e^{\mathrm{i}\omega t}
\end{equation*}
Agar dapat menjadi solusi, kita memerlukan
\begin{equation*}
C = \frac{f_{0}}{\omega_{0}^{2} - \omega^{2} + 2\mathrm{i}\beta\omega}
\end{equation*}
Tulis koefisien $C$ (yang merupakan bilangan kompleks) dalam bentuk
\begin{equation*}
C = Ae^{-\mathrm{i}\delta}
\end{equation*}
di mana $A$ dan $\delta$ adalah bilangan real.

Nilai $A$ dapat diperoleh dari
\begin{align*}
A^{2} & = CC^{*} =
\frac{f_{0}}{\omega_{0}^{2} - \omega^{2} + 2\mathrm{i}\beta\omega} \cdot 
\frac{f_{0}}{\omega_{0}^{2} - \omega^{2} - 2\mathrm{i}\beta\omega} \\
 & = \frac{f_{0}^{2}}{\left(\omega_{0}^{2} - \omega^{2}\right)^{2} + 4\beta^{2}\omega^{2}}
\end{align*}
Untuk nilai $f_0$ dan $\beta$ tetap, amplitudo $A$ akan
bernilai paling besar ketika $\omega_{0} \approx \omega$.
Dengan kata lain, osilator
akan merespon paling baik ketika diberikan gaya periodik dengan
frekuensi $\omega$ yang dekat dengan frekuensi natural dari osilator.
Fenomena ini dikenal dengan sebagai \emph{resonansi}.

Untuk menemukan sudut fasa $\delta$ kita dapat menggunakan
\begin{equation*}
C = \frac{f_{0}}{\omega_{0}^{2}-\omega^{2}+2\mathrm{i}\beta\omega}
\end{equation*}
dan
\begin{equation*}
C = Ae^{-\mathrm{i}\delta}
\end{equation*}
Dari dua ekspresi untuk $C$ di atas, dapat diperoleh
\begin{equation*}
\frac{f_{0}}{\omega_{0}^{2} - \omega^{2} + 2\mathrm{i}\beta\omega} = A e^{-\mathrm{i}\delta}
\end{equation*}
atau
\begin{equation*}
f_{0}e^{\mathrm{i}\delta} = A\left(\omega_{0}^{2} - \omega^{2} + 2\mathrm{i}\beta\omega\right)
\end{equation*}
Karena $f_{0}$ dan $A$ sama-sama bernilai real, maka sudut fasa
$\delta$ sama dengan sudut fasa dari bilangan kompleks $\omega_{0}^{2}-\omega^{2} + 2\mathrm{i}\beta\omega$,
yaitu
\begin{equation*}
\delta=\arctan\left(\frac{2\beta\omega}{\omega_{0}^{2}-\omega^{2}}\right)
\end{equation*}

Karena $A$ dan $\delta$ sudah dapat diperoleh, kita dapat membentuk solusi kompleks
$z(t)$ yang diperlukan, yaitu
\begin{equation}
z(t) = Ce^{\mathrm{i}\omega t} = A e^{\mathrm{i}(\omega t-\delta)}
\end{equation}
dan bagian real dari fungsi ini adalah solusi yang kita cari, yaitu:
\begin{equation}
x(t) = A\cos(\omega t - \delta)
\end{equation}

Solusi ini hanyalah satu solusi khusus dari persamaan gerak. Solusi
umum diperoleh dengan menambahkan sembarang solusi dari persamaan
homogon terkait, sehingga diperoleh solusi umum sebagai berikut
\begin{equation}
x(t) = A\cos(\omega t-\delta) + C_{1}e^{r_{1}t} + C_{2}e^{r_{2}t}
\end{equation}
Karena dua suku ekstra ada solusi umum ini merupakan eksponensial
yang meluruh (dengan asumsi bagian real dari $r_{1}$ dan $r_{2}$
adalah bilangan negatif), maka suku-suku ini disebut sebagai suku
transien. Suku ini bergantung pada syarat awal, namum pada akhirnya
akan tidak relevan untuk perilaku solusi pada jangka panjang. Oleh
karena itu, suku cosinus (suku pertama) pada akhirnya akan mendominasi.

Untuk kasus osilasi teredam, dua suku terakhir dapat ditulis sebagai
cosinus meluruh dengan pergeseran fasa:
\begin{equation}
x(t)=A\cos(\omega t-\delta)+A_{\mathrm{tr}}e^{-\beta t}\cos(\omega_{1}-\delta_{\mathrm{tr}})
\end{equation}